% % Apunte de modulos basicos
%
\documentclass[a4paper,10pt]{article}
\usepackage[paper=a4paper, hmargin=1.5cm, bottom=1.5cm, top=3.5cm]{geometry}
\usepackage[latin1]{inputenc}
\usepackage[T1]{fontenc}
\usepackage[spanish]{babel}
\usepackage{fancyhdr}
\usepackage{lastpage}
\usepackage{xspace}
\usepackage{xargs}
\usepackage{ifthen}
\usepackage{aed2-tad,aed2-symb,aed2-itef}
\usepackage{algorithmicx, algpseudocode, algorithm}
\usepackage[colorlinks=true, linkcolor=blue]{hyperref}


\hypersetup{%
 % Para que el PDF se abra a p�gina completa.
 pdfstartview= {FitH \hypercalcbp{\paperheight-\topmargin-1in-\headheight}},
 pdfauthor={C�tedra de Algoritmos y Estructuras de Datos II - DC - UBA},
 pdfkeywords={M�dulos b�sicos},
 pdftitle={M�dulos b�sicos de dise�o},
 pdfsubject={M�dulos b�sicos de dise�o}
}

%%%%%%%%%%%%%%%%%%%%%%%%%%%%%%%%%%%%%%%%%%%%%%%%%
% PARAMETROS A SER MODIFICADOS
%%%%%%%%%%%%%%%%%%%%%%%%%%%%%%%%%%%%%%%%%%%%%%%%%

%cuatrimestre de acuerdo a la opcion
\newcommand{\Cuatrimestre}{$2^\mathrm{do}$ cuatrimestre de 2021}

%%%%%%%%%%%%%%%%%%%%%%%%%%%%%%%%%%%%%%%%%%%%%%%%%
% OTRAS OPCIONES QUE NO HAY QUE MODIFICAR
%%%%%%%%%%%%%%%%%%%%%%%%%%%%%%%%%%%%%%%%%%%%%%%%%

%opening
\title{Resumen SQL}
\author{Programacion Frontend}
\date{Julio 2023}

% Acomodo fancyhdr.
\pagestyle{fancy}
\thispagestyle{fancy}
\lhead{Resumen SQL}
\rhead{Julio 2023}
\cfoot{\thepage /\pageref{LastPage}}
\renewcommand{\footrulewidth}{0.4pt}
\setlength{\headheight}{13pt}

%%%%%%%%%%%%%%%%%%%%%%%%%%%%%%%%%%%%%%%%%%%%%%%%%%%%%%%%%%%%
% COMANDOS QUE ALGUN DIA PUEDAN FORMAR UN PAQUETE.
%%%%%%%%%%%%%%%%%%%%%%%%%%%%%%%%%%%%%%%%%%%%%%%%%%%%%%%%%%%%
\newcommand{\moduloNombre}[1]{\textbf{#1}}

\let\NombreFuncion=\textsc
\let\TipoVariable=\texttt
\let\ModificadorArgumento=\textbf
\newcommand{\res}{$res$\xspace}
\newcommand{\tab}{\hspace*{7mm}}

\newcommandx{\TipoFuncion}[3]{%
  \NombreFuncion{#1}(#2) \ifx#3\empty\else $\to$ \res\,: \TipoVariable{#3}\fi%
}
\newcommand{\In}[2]{\ModificadorArgumento{in} \ensuremath{#1}\,: \TipoVariable{#2}\xspace}
\newcommand{\Out}[2]{\ModificadorArgumento{out} \ensuremath{#1}\,: \TipoVariable{#2}\xspace}
\newcommand{\Inout}[2]{\ModificadorArgumento{in/out} \ensuremath{#1}\,: \TipoVariable{#2}\xspace}
\newcommand{\Aplicar}[2]{\NombreFuncion{#1}(#2)}

\newlength{\IntFuncionLengthA}
\newlength{\IntFuncionLengthB}
\newlength{\IntFuncionLengthC}
%InterfazFuncion(nombre, argumentos, valor retorno, precondicion, postcondicion, complejidad, descripcion, aliasing)
\newcommandx{\InterfazFuncion}[9][4=true,6,7,8,9]{%
  \hangindent=\parindent
  \TipoFuncion{#1}{#2}{#3}\\%
  \textbf{Pre} $\equiv$ \{#4\}\\%
  \textbf{Post} $\equiv$ \{#5\}%
  \ifx#6\empty\else\\\textbf{Complejidad:} #6\fi%
  \ifx#7\empty\else\\\textbf{Descripci�n:} #7\fi%
  \ifx#8\empty\else\\\textbf{Aliasing:} #8\fi%
  \ifx#9\empty\else\\\textbf{Requiere:} #9\fi%
}

\newenvironment{Interfaz}{%
  \parskip=2ex%
  \noindent\textbf{\Large Interfaz}%
  \par%
}{}

\newenvironment{Representacion}{%
  \vspace*{2ex}%
  \noindent\textbf{\Large Representaci�n}%
  \vspace*{2ex}%
}{}

\newenvironment{Algoritmos}{%
  \vspace*{2ex}%
  \noindent\textbf{\Large Algoritmos}%
  \vspace*{2ex}%
}{}


\newcommand{\Titulo}[1]{
  \vspace*{1ex}\par\noindent\textbf{\large #1}\par
}

\newenvironmentx{Estructura}[2][2={estr}]{%
  \par\vspace*{2ex}%
  \TipoVariable{#1} \textbf{se representa con} \TipoVariable{#2}%
  \par\vspace*{1ex}%
}{%
  \par\vspace*{2ex}%
}%

\newboolean{EstructuraHayItems}
\newlength{\lenTupla}
\newenvironmentx{Tupla}[1][1={estr}]{%
    \settowidth{\lenTupla}{\hspace*{3mm}donde \TipoVariable{#1} es \TipoVariable{tupla}$($}%
    \addtolength{\lenTupla}{\parindent}%
    \hspace*{3mm}donde \TipoVariable{#1} es \TipoVariable{tupla}$($%
    \begin{minipage}[t]{\linewidth-\lenTupla}%
    \setboolean{EstructuraHayItems}{false}%
}{%
    $)$%
    \end{minipage}
}

\newcommandx{\tupItem}[3][1={\ }]{%
    %\hspace*{3mm}%
    \ifthenelse{\boolean{EstructuraHayItems}}{%
        ,#1%
    }{}%
    \emph{#2}: \TipoVariable{#3}%
    \setboolean{EstructuraHayItems}{true}%
}

\newcommandx{\RepFc}[3][1={estr},2={e}]{%
  \tadOperacion{Rep}{#1}{bool}{}%
  \tadAxioma{Rep($#2$)}{#3}%
}%

\newcommandx{\Rep}[3][1={estr},2={e}]{%
  \tadOperacion{Rep}{#1}{bool}{}%
  \tadAxioma{Rep($#2$)}{true \ssi #3}%
}%

\newcommandx{\Abs}[5][1={estr},3={e}]{%
  \tadOperacion{Abs}{#1/#3}{#2}{Rep($#3$)}%
  \settominwidth{\hangindent}{Abs($#3$) \igobs #4: #2 $\mid$ }%
  \addtolength{\hangindent}{\parindent}%
  Abs($#3$) \igobs #4: #2 $\mid$ #5%
}%

\newcommandx{\AbsFc}[4][1={estr},3={e}]{%
  \tadOperacion{Abs}{#1/#3}{#2}{Rep($#3$)}%
  \tadAxioma{Abs($#3$)}{#4}%
}%


\newcommand{\DRef}{\ensuremath{\rightarrow}}

\begin{document}

%pagina de titulo
\thispagestyle{empty}
\maketitle
\tableofcontents

\newpage


\section{Clausula SELECT}

\TipoVariable{SELECT * FROM users} se utiliza para seleccionar todo desde la tabla \TipoVariable{users}. Ahora, vamos a insertar en la tabla users los campos (columnas) name, surname y age de la siguiente manera:
\newline
\newline
\TipoVariable{INSERT INTO users (name,surname,age)  VALUES ('Ezequiel', 'Rueda Sanchez', '24')}
\newline
\newline
Observamos que al ejecutar esta instrucci�n obtenemos:
\newline
\newline
\TipoVariable{'Result: query executed successfully. Took 5ms'}.
\newline
\newline
Esto es as� porque cualquier operaci�n (en este caso \TipoVariable{INSERT}) que realicemos con SQL la toma como una consulta.
\newline
\newline
Ahora si ejecutamos \TipoVariable{SELECT * FROM users} nos devuelve la siguiente tabla:
\newline
\newline
\begin{tabular}{|c|c|c|}
	\hline
	Name     &  Surname         &   Age     \\ \hline
	Ezequiel &  Rueda Sanchez   &   24      \\ \hline
\end{tabular}
\newline
\newline
\newline
Tambi�n puedo agregar varios usuarios al mismo tiempo de la siguiente manera:
\newline
\newline
\TipoVariable{INSERT into users (name, surname,  age) VALUES ('Pedro', 'Rodriguez', 26), ('Carlos', 'Sanchez', 18), ('Francisco', 'Alvarez', 52)}
\newline
\newline
Ahora, ejecuto \TipoVariable{SELECT * FROM users} para ver el nuevo estado de la tabla.
\newline
\newline
Al ejecutar \TipoVariable{SELECT age FROM users}, filtramos la tabla para que solo nos devuelva el campo \TipoVariable{age} con sus respectivos registros.
\newline
\newline
Ahora insertamos el registro \TipoVariable{('Francisco', 'Alvarez', 52)} y tenemos dos registros iguales. 
\newline
\newline
�Como los podemos diferenciar?
\newline
\newline
No hay manera de diferenciarlos. Para eso, utilizamos los \textbf{identificadores}. Tenemos los $primary$ $keys$ y los $foreign$ $keys$.
\newline
\newline
Para ello, vamos a modificar la tabla colocando un nuevo campo llamado \TipoVariable{id\_user} de tipo integer. 
\newline
\newline
Ahora, al mostrar la tabla (\TipoVariable{SELECT * FROM users}), observamos que todos los registros tienen el campo \TipoVariable{id\_user} con \TipoVariable{NULL}.
\newline
\newline
Entonces, vamos a borrar todos los registros ejecutando \TipoVariable{DELETE FROM users}.
\newline
\newline
Ahora, vuelvo a modificar la tabla y puedo aplicar la opci�n \TipoVariable{AI} (Auto Increment) para que cada registro sea unico y autom�ticamente se me aplica la opci�n \TipoVariable{PK} (Primary Key) porque nos garantiza una integridad referencial, es decir, podemos hacer referencias a registros sin tener valores duplicados.
\newline
\newline
Por lo tanto, ahora a medida que vayamos insertando registros el id se va a auto incrementar en uno para que no haya ninguna posibilidad de que dos registros sean exactamente iguales. Luego, armamos una tabla nueva que llamamos \TipoVariable{turnos\_medicos} donde insertamos los campos \TipoVariable{id\_turno}, \TipoVariable{profesional}, \TipoVariable{id\_user} (F(oreign) K(ey) o clave foranea), \TipoVariable{motivo} y \TipoVariable{horario}.
\newline
\newline
Decimos que una clave es for�nea en una tabla cuando hace referencia a una clave primaria (PK) de otra tabla y se le debe colocar el mismo nombre tanto en la tabla como en la tabla donde es PK.
\newline
\newline
De ahora en adelante, vamos a trabajar con la base de datos de \textbf{Northwind}.
\newline
\newline
\newline
Supongamos que queremos cambiar el nombre del campo \TipoVariable{LastName}. No lo puedo hacer desde la tabla pero puedo modificarlo localmente con \TipoVariable{SELECT LastName AS surname FROM Employees}.
\newline
\newline
De ahora en adelante, esta renombrado localmente como \TipoVariable{surname} aunque en la tabla el campo sigue siendo \TipoVariable{LastName}.
\newline
\newline
Tambi�n podemos renombrarlo varios campos mediante comas. Por ejemplo: 
\newline
\newline
\TipoVariable{SELECT LastName AS surname, FirstName AS name  FROM Employees}
\newline
\newline
Por lo tanto, utilizamos la cl�usula \TipoVariable{AS} para que una tabla/campo sea mas declarativo o descriptivo.
\newline
\newline
Ahora, volvemos a ver toda la de products ejecutando \TipoVariable{SELECT * FROM Products}.
\newline
\newline
Luego, quiero ordenar la tabla pero seg�n el precio (campo \TipoVariable{PRICE}) de forma creciente. Sabemos que por defecto, los productos est�n ordenados seg�n el numero de \TipoVariable{ID} porque son autoincrementables. Para ello ejecutamos las siguientes lineas:
\newline
\newline
\TipoVariable{SELECT * FROM Products ORDER BY price}
\newline
\newline
Tambi�n podr�amos ejecutar \TipoVariable{SELECT * FROM Products ORDER BY price ASC}. Y  obtenemos el mismo resultado.
\newline
\newline
Si queremos ordenar los precios de forma decreciente, ejecutamos las siguientes lineas:
\newline
\newline
\TipoVariable{SELECT * FROM Products ORDER BY price DESC}
\newline
\newline
\TipoVariable{ASC} y \TipoVariable{DESC} funcionan tanto para los precios como para el campo text \TipoVariable{ProductName}. En este caso, por defecto o \TipoVariable{ASC} ordena alfabeticamente de menor a mayor (de la A a la Z) y \TipoVariable{DESC} ordena alfabeticamente de mayor a menor (de la Z a la A).
\newline
\newline
Notemos que la tabla ordena de acuerdo a la siguiente jerarqu�a:
\newline
\newline
\TipoVariable{NULL $>$ NUMEROS $>$ CARACTERES ESPECIALES $>$ CARACTERES COMUNES}
\newline
\newline
Por lo tanto, si agregamos un registro \TipoVariable{NULL} en la tabla de \TipoVariable{Products} y ejecutamos:
\newline
\newline
\TipoVariable{SELECT * FROM Products ORDER BY ProductName ASC}
\newline
\newline
El registro \TipoVariable{NULL} se convierte en el primero de la tabla.
\newline
\newline
Ahora, si quiero colocar todos los registros NULL al final de la tabla ejecuto:
\newline
\newline
\TipoVariable{SELECT * FROM Products ORDER BY ProductName ASC NULLS LAST}
\newline
\newline
\TipoVariable{ASC} esta ordenando de forma creciente, es decir, los \TipoVariable{NULLs} van al principio pero al ejecutar \TipoVariable{NULLs LAST} estoy indicando que los voy a colocar al final de la tabla.
\newline
\newline
Ahora, si quiero colocar todos los registros NULL al principio de la tabla cuando estoy ordenando de forma descendente ejecuto:
\newline
\newline
\TipoVariable{SELECT * FROM Products ORDER BY ProductName DESC NULLS FIRST}
\newline
\newline
\TipoVariable{ASC} esta ordenando de forma creciente, es decir, los \TipoVariable{NULLs} van al principio pero al ejecutar \TipoVariable{NULLs LAST} estoy indicando que los voy a colocar al final de la tabla.
\newline
\newline
Ahora, si queremos ordenar la tabla de una manera \TipoVariable{RANDOM} ejecutamos:
\newline
\newline
\TipoVariable{SELECT * FROM Products ORDER BY RANDOM()}
\newline
\newline
Ahora, si ejecutamos:
\newline
\newline
\TipoVariable{SELECT * FROM Products ORDER BY ProductName, SupplierID DESC}
\newline
\newline
Se ordena primero respecto a \TipoVariable{ProductName} y en caso de que los registros sean iguales, en base a \TipoVariable{SupplierID}. Ahora, si queremos eliminar todos los repetidos por ejemplo de \TipoVariable{ProductName} ejecutamos:
\newline
\newline
\TipoVariable{SELECT DISTINCT ProductName FROM Products}
\newline
\newline
Con esto eliminamos todas las repeticiones de \TipoVariable{NULL} y solo nos quedamos con una (con la primera aparici�n).
\newline
\newline
Ademas, podemos ordenar los registros del campo ProductName ejecutando lo siguiente:
\newline
\newline
\TipoVariable{SELECT DISTINCT ProductName FROM Products ORDER BY ProductName DESC}
\newline
\newline
Por lo tanto, primero solo se deja la primera aparici�n de cada producto y luego se ordena los mismos de manera decreciente.
\newpage

\section{Clausula WHERE}

Supongamos que queremos ver cual es el nombre asociado al \TipoVariable{productId} numero 14. Entonces, comenzamos ejecutando:
\newline
\newline
\TipoVariable{SELECT ProductName FROM Products} obteniendo el listado de productos.
\newline
\newline
Ahora si queremos ver cual es el nombre asociado al productId numero 14 ejecutamos:
\newline
\newline
\TipoVariable{SELECT ProductName FROM Products WHERE ProductID = 14}
\newline
\newline
Si queremos obtener toda la informaci�n del registro con \TipoVariable{ProductID = 14} ejecutamos:
\newline
\newline
\TipoVariable{SELECT * FROM Products WHERE ProductID = 14}
\newline
\newline
Ahora, supongamos que solo tenemos un presupuesto de 40USD y quiero ver que productos puedo comprar. Para ello, ejecuto lo siguiente:
\newline
\newline
\TipoVariable{SELECT * FROM Products WHERE Price $<$ 40}
\newline
\newline
Ahora, supongamos que queremos eliminar el registro \TipoVariable{NULL} con \TipoVariable{ProductID = 80}. Para ello, ejecutamos lo siguiente:
\newline
\newline
\TipoVariable{DELETE FROM Products WHERE ProductID = 80}
\newline
\newline
Observemos que no es necesario escribir \TipoVariable{*} luego del \TipoVariable{DELETE}.



\end{document}
